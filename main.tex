\documentclass[12pt]{article}
\usepackage{fontspec}
\usepackage{xcolor}
\usepackage{listings}
\usepackage[justification=centering]{caption}
\usepackage{subcaption}
\usepackage{cite}
\usepackage{amssymb}
\usepackage{amsmath}
\usepackage{amsthm}
\usepackage{mathtools}
\usepackage{bussproofs}
\usepackage{tikz}
\usepackage{float}
\usepackage{graphicx}
\usepackage{algorithm}
\usepackage[noend]{algpseudocode}
\usepackage{array}
\usepackage{stmaryrd}
\usepackage{pifont}

\newcommand{\rulehskip}{\hskip 1.5em}
\newcommand{\rulevspace}{\vspace{1em}}

\mathchardef\mhyphen="2D

\theoremstyle{definition}

\newtheorem{definition}{Definition}
\newtheorem{theorem}{Theorem}
\newtheorem{lemma}{Lemma}
\newtheorem{prop}{Proposition}

%% \counterwithin{lemma}{section}

\newcommand{\textdef}[1]{\textit{#1}}

\newcommand{\imm}{{\textrm IMM}~}

% inline code 
\newcommand{\code}[1]{\texttt{#1}}

% tuple with angle brackets
\newcommand{\tup}[1]{\langle #1 \rangle}

% semantics brackets
\newcommand{\sem}[1]{\llbracket #1 \rrbracket}

% equality by definition
\newcommand{\defeq}{\triangleq}

% function arrow
\newcommand{\fun}{\rightarrow}

% partial function arrow
\newcommand{\pfun}{\rightharpoonup}

% some math sets
\newcommand{\N}{{\mathbb{N}}}
\newcommand{\Q}{{\mathbb{Q}}}

% domain/codomain notation
\newcommand{\dom}[1]{\textit{dom}{({#1})}}
\newcommand{\codom}[1]{\textit{codom}{({#1})}}

\newcommand{\lca}{\textit{lca}}

% some logical notation
%\newcommand{\implies}{{\Rightarrow}}
%\newcommand{\iff}{{\Leftrightarrow}}

% check-mark and cross-mark
\newcommand{\cmark}{\text{\color{green}\ding{51}}}
\newcommand{\xmark}{\text{\color{red}\ding{55}}}

% event structure
\newcommand{\lES}{S}
% event structure's execution

% execution graph
\newcommand{\lG}{G}
% covered/issued events
\newcommand{\lC}{C}
\newcommand{\lI}{I}
% mapping between events
\newcommand{\lM}{m}
\newcommand{\lMC}{m_{c}}
\newcommand{\lMI}{m_{i}}

% event set
\newcommand{\lE}{{\mathtt{E}}}
\newcommand{\lEi}[1]{\lE_{#1}}
\newcommand{\lEo}{\lEi{0}}
\newcommand{\lEs}{\widehat{{\mathtt{E}}}}
\newcommand{\lEsi}[1]{\lEs_{#1}}
\newcommand{\lEso}{\lEsi{0}}
\newcommand{\lR}{{\mathtt{R}}}
\newcommand{\lW}{{\mathtt{W}}}
\newcommand{\lF}{{\mathtt{F}}}
\newcommand{\lX}{{\mathtt{X}}}

\newcommand{\lEsqeq}[1]{{\lE^{\sqsupseteq#1}}}
\newcommand{\lRsqeq}[1]{{\lR^{\sqsupseteq#1}}}
\newcommand{\lWsqeq}[1]{{\lW^{\sqsupseteq#1}}}
\newcommand{\lFsqeq}[1]{{\lF^{\sqsupseteq#1}}}

% labeling of events
\newcommand{\Tid}{\mathsf{Tid}}
\newcommand{\Loc}{\mathsf{Loc}}
\newcommand{\Val}{\mathsf{Val}}
\newcommand{\Bool}{\mathsf{Bool}}
\newcommand{\Lab}{\mathsf{Lab}}
\newcommand{\Mod}{\mathsf{Mod}}
\newcommand{\Cont}{\kappa}
\newcommand{\Scp}{\mathsf{Scope}}
\newcommand{\Path}{\pi}
\newcommand{\Prog}{\mathsf{prog}}
\newcommand{\Execs}[1]{\Psi_{#1}}

\newcommand{\lLAB}{{\mathtt{lab}}}
\newcommand{\lTID}{{\mathtt{tid}}}
\newcommand{\lTYP}{{\mathtt{typ}}}
\newcommand{\lLOC}{{\mathtt{loc}}}
\newcommand{\lMOD}{{\mathtt{mod}}}
\newcommand{\lSCP}{{\mathtt{scp}}}
\newcommand{\lCONT}{{\mathtt{K}}}
\newcommand{\lVALR}{{\mathtt{val_r}}}
\newcommand{\lVALW}{{\mathtt{val_w}}}
\newcommand{\lINIT}{{\mathtt{init}}}
\newcommand{\lSN}{{\mathtt{sn}}}
\newcommand{\lPC}{{\mathtt{pc}}}
\newcommand{\lPREV}{{\mathtt{prev}}}

\newcommand{\lMASK}{{\mathtt{Mask}}}
\newcommand{\lVIS}{{\mathtt{Vis}}}

% simulation relation 
\newcommand{\simR}{\mathcal{I}}
\newcommand{\simRb}{\mathcal{I}_{base}}
\newcommand{\simRw}{\mathcal{I}_{weak}}

\newcommand{\nextset}{{\mathtt{Next}}}
\newcommand{\coverable}{{\mathtt{Coverable}}}
\newcommand{\issuable}{{\mathtt{Issuable}}}
\newcommand{\front}{\phi}
\newcommand{\forward}{{\mathtt{Forward}}}
\newcommand{\cert}{{\mathtt{cert}}}
\newcommand{\certify}{{\mathtt{Certify}}}

%\newcommand{\lCERTIFY}{{\mathtt{Certify}}}

\newcommand{\trstep}{\longrightarrow}
\newcommand{\esbstep}[1]{\xLongrightarrow{#1}}
\newcommand{\esstep}[1]{\xlongrightarrow{#1}}
\newcommand{\esstepstar}{\longrightarrow^*}

\newcommand{\ESle}{\preceq}
\newcommand{\ESbasicstep}[1]{\xLongrightarrow{#1}}
\newcommand{\ESstep}[1]{\xlongrightarrow{#1}}
\newcommand{\ESstepstar}{\longrightarrow^*}

\newcommand{\rlab}[3]{\lR^{#1}({#2},{#3})}
\newcommand{\wlab}[3]{\lW^{#1}({#2},{#3})}
\newcommand{\flab}[1]{\lF^{#1}}

\newcommand{\Ex}{{\mathtt{ex}}}
\newcommand{\NotEx}{{\mathtt{not \mhyphen ex}}}
\newcommand{\rlabExExpl}[4]{\lR_{#4}^{#1}({#2},{#3})}
\newcommand{\rlabEx}[3]{\rlabExExpl{#1}{#2}{#3}{\Ex}}

%% \newcommand{\rlabS}[3]{\lR^{#3}({#1},{#2})}
%% \newcommand{\wlabS}[3]{\lW^{#3}({#1},{#2})}
%% \newcommand{\flabS}[1]{\lF^{#1}}

% some predefined relations
\colorlet{colorPO}{gray!60!black}
\colorlet{colorPPO}{magenta}
\colorlet{colorRF}{green!60!black}
\colorlet{colorJF}{green!40!black}
\colorlet{colorRMW}{olive!70!black}
\colorlet{colorCF}{red!60!black}
\colorlet{colorEW}{brown}
\colorlet{colorMO}{orange!60!black}
\colorlet{colorRB}{purple}
\colorlet{colorECO}{red!80!black}
\colorlet{colorRSEQ}{cyan}
\colorlet{colorRELP}{cyan}
\colorlet{colorSW}{blue!40!black}
\colorlet{colorHB}{blue}
\colorlet{colorSCB}{violet}
\colorlet{colorPSC}{violet}
\colorlet{colorFSC}{violet}
\colorlet{colorSC}{violet}
\colorlet{colorWGR}{magenta!40!black}
\colorlet{colorDEV}{magenta}
\colorlet{colorINCL}{magenta!70!black}
\colorlet{colorOBS}{green!40!black}
\colorlet{colorCA}{teal}
\colorlet{colorDEPS}{violet!60!black}
\colorlet{colorDETOUR}{teal}
\colorlet{colorBOB}{gray!80!black}

\newcommand{\lPO}{{\color{colorPO}\mathtt{po}}}
\newcommand{\lPPO}{{\color{colorPPO}\mathtt{ppo}}}
\newcommand{\lRF}{{\color{colorRF}\mathtt{rf}}}
\newcommand{\lJF}{{\color{colorRF}\mathtt{jf}}}
\newcommand{\lRMW}{{\color{colorRMW}\mathtt{rmw}}}
\newcommand{\lCF}{{\color{colorCF}\mathtt{cf}}}
\newcommand{\lEW}{{\color{colorEW}\mathtt{ew}}}
\newcommand{\lMO}{{\color{colorMO}\mathtt{mo}}}
\newcommand{\lRB}{{\color{colorRB}\mathtt{rb}}}
\newcommand{\lECO}{{\color{colorECO}\mathtt{eco}}}

\newcommand{\lRSEQ}{{\color{colorRSEQ}\mathtt{rseq}}}
\newcommand{\lRELP}{{\color{colorRELP}\mathtt{relp}}}
\newcommand{\lSW}{{\color{colorSW}\mathtt{sw}}}
\newcommand{\lHB}{{\color{colorHB}\mathtt{hb}}}

\newcommand{\ocl}{{\textrm OpenCL}}
\newcommand{\ptx}{{\textrm PTX}}
\newcommand{\lSWocl}{{\color{colorSW}\mathtt{sw_\ocl}}}
\newcommand{\lSWptx}{{\color{colorSW}\mathtt{sw_\ptx}}}

\newcommand{\lSCB}{{\color{colorSCB} \mathtt{scb}}}
\newcommand{\lPSCB}{\lPSC_{\rm base}}
\newcommand{\lPSCF}{\lPSC_\lF}
\newcommand{\lFSC}{{\color{colorFSC}\mathtt{fsc}}}
\newcommand{\lPSC}{{\color{colorPSC}\mathtt{psc}}}
\newcommand{\lSC}{{\color{colorSC}\mathtt{sc}}}

\newcommand{\lWGR}{{\color{colorWGR}\mathtt{wgr}}}
\newcommand{\lWGRi}[1]{{\color{colorWGR}\mathtt{wgr}_{#1}}}
\newcommand{\lDEV}{{\color{colorDEV}\mathtt{dev}}}
\newcommand{\lINCL}{{\color{colorINCL}\mathtt{incl}}}
\newcommand{\lHBINCL}{\lHB_{\lINCL}}

\newcommand{\lOBS}{{\color{colorOBS}\mathtt{obs}}}
\newcommand{\lCA}{{\color{colorCA}\mathtt{ca}}}
\newcommand{\lCAB}{\lCA_{\rm base}}

\newcommand{\lCTRL}{{{\color{colorDEPS}\mathtt{ctrl}}}}
\newcommand{\lDATA}{{{\color{colorDEPS}\mathtt{data}}}}
\newcommand{\lADDR}{{{\color{colorDEPS}\mathtt{addr}}}}
\newcommand{\lCASDEP}{{{\color{colorDEPS}\mathtt{casdep}}}}
\newcommand{\lDEPS}{{\color{colorDEPS}\mathtt{dep}}}

\newcommand{\lBOB}{{\color{colorBOB}\mathtt{bob}}}
\newcommand{\lDETOUR}{{\color{colorDETOUR}\mathtt{detour}}}

\newcommand{\lmakeE}[1]{#1\mathtt{e}}
\newcommand{\lRFE}{\lmakeE{\lRF}}
\newcommand{\lCOE}{\lmakeE{\lCO}}
\newcommand{\lRBE}{\lmakeE{\lRB}}
\newcommand{\lMOE}{\lmakeE{\lMO}}

\newcommand{\lmakeI}[1]{#1\mathtt{i}}
\newcommand{\lRFI}{\lmakeI{\lRF}}
\newcommand{\lCOI}{\lmakeI{\lCO}}
\newcommand{\lFRI}{\lmakeI{\lFR}}

\newcommand{\lmakeLoc}[1]{#1_{|loc}}

%% memory orders
\newcommand{\na}{\mathtt{na}}
\newcommand{\pln}{\mathtt{pln}}
\newcommand{\rlx}{\mathtt{rlx}}
\newcommand{\rel}{{\mathtt{rel}}}
\newcommand{\acq}{{\mathtt{acq}}}
\newcommand{\con}{{\mathtt{con}}}
\newcommand{\acqrel}{{\mathtt{acqrel}}}
\newcommand{\relAcq}{\acqrel}
\newcommand{\sco}{{\mathtt{sc}}}

%% scopes
\newcommand{\wgr}{\mathtt{wgr}}
\newcommand{\dev}{\mathtt{dev}}
\newcommand{\sys}{\mathtt{sys}}

%% tikz stuff

\newcommand{\event}[3]{#1#2#3}
\tikzset{
   every path/.style={>=stealth},
   po/.style={->,color=colorPO,,shorten >=-0.5mm,shorten <=-0.5mm},
   rf/.style={->,color=colorRF,dashed,,shorten >=-0.5mm,shorten <=-0.5mm},
   rb/.style={->,color=colorRB,thick,shorten >=-0.5mm,shorten <=-0.5mm},
   mo/.style={->,color=colorMO,dotted,thick,shorten >=-0.5mm,shorten <=-0.5mm},
   sw/.style={->,color=colorSW,dashed,thick,shorten >=-0.5mm,shorten <=-0.5mm},
   obs/.style={->,color=colorOBS,dashed,,shorten >=-0.5mm,shorten <=-0.5mm},
   no/.style={->,dotted,thick,shorten >=-0.5mm,shorten <=-0.5mm},
   deps/.style={->,color=colorDEPS,dotted,thick,shorten >=-0.5mm,shorten <=-0.5mm},
   wgr/.style={fill=colorWGR, opacity=0.1}
}

%% parallel threads listings

\newcommand{\inarr}[1]{\begin{array}{@{}l@{}}#1\end{array}}
\newcommand{\inarrII}[2]{\begin{array}{@{}l@{~~}||@{~~}l@{}}\inarr{#1}&\inarr{#2}\end{array}}
\newcommand{\inarrIII}[3]{\begin{array}{@{}l@{~~}||@{~~}l@{~~}||@{~~}l@{}}\inarr{#1}&\inarr{#2}&\inarr{#3}\end{array}}
\newcommand{\inarrIV}[4]{\begin{array}{@{}l@{~~}||@{~~}l@{~~}||@{~~}l@{~~}||@{~~}l@{}}\inarr{#1}&\inarr{#2}&\inarr{#3}&\inarr{#4}\end{array}}
\newcommand{\inarrV}[5]{\begin{array}{@{}l@{~~}||@{~~}l@{~~}||@{~~}l@{~~}||@{~~}l@{~~}||@{~~}l@{}}\inarr{#1}&\inarr{#2}&\inarr{#3}&\inarr{#4}&\inarr{#5}\end{array}}

%% instructions for listings

\newcommand{\valueRead}[1]{{// #1}}

\newcommand{\fence}{\texttt{fence}}
\newcommand{\readInstS}[3]{#2 \;:=^{#1}\;[#3]}
\newcommand{\writeInstS}[3]{[#2] \;:=^{#1}\;#3}
\newcommand{\fenceInstS}[1]{\fence^{#1}}

%% axiom labels

\newcounter{mylabelcounter}

\makeatletter
\newcommand{\labelAxiom}[2]{%
\hfill{\normalfont\textsc{(#1)}}\refstepcounter{mylabelcounter}
\immediate\write\@auxout{%
  \string\newlabel{#2}{{\unexpanded{\normalfont\textsc{#1}}}{\thepage}{{\unexpanded{\normalfont\textsc{#1}}}}{mylabelcounter.\number\value{mylabelcounter}}{}}
}%
}
\makeatother

%% warning

\colorlet{colorWARNING}{yellow!90!black}

\newcommand{\warning}[1]{{\color{colorWARNING}\texttt{WARNING}}: #1}
\newcommand{\app}[1]{{\color{blue}\textbf{ANTON: #1}}}
\newcommand{\note}[1]{{\color{cyan}\textbf{EVG: #1}}}
\newcommand{\todo}[1]{{\color{red}\textbf{TODO: #1}}}


\begin{document}

\begin{center}
{\center \LARGE Compiling Event Structures to the Intermediate Memory Model }
\end{center}

\section{Event Structure Model}

\begin{definition}
  An \emph{event structure} $\lES$ is a tuple
  $\tup{\lE, \lEi, \lTID, \lLAB, \lMOD, \lPO, \lRMW, \lRF, \lEW, \lMO}$ where:
  \begin{itemize}
  \item $\lE \subseteq \N$ --- finite set of events. 
  \item $\lEi \subseteq \lE$ --- initialization events.
  \item $\lTID \defeq \fun{\lE}{\Tid}$ --- function that assigns a thread id to every event.
    Given thread id $i \in \Tid$ we will denote by $\lE_i$ the set of all events belonging 
    to $i^{th}$ thread, that is $\lE_i \defeq \{e \in \lE ~|~ \lTID(e) = i\}$.
    We assume that if $e \in \lEi$ then $\lTID(e) = 0$.
  \item $\lLAB \defeq \fun{\lE}{\Lab}$ --- function that assigns a label to every event.
    Labels are of one of the following forms:
    \begin{itemize}
    \item $\rlab{x}{v}^s$ --- a read, where $x \in \Loc$, $v \in \Val$ 
      and $s \in \Bool$ is an exclusive flag 
      (we will denote exclusive reads as $\rlabEx{x}{v}$);
    \item $\wlab{x}{v}$ --- a write, where $x \in \Loc$, $v \in \Val$;
    \item $\flab$ --- a fence.
    \end{itemize}
    $\lLAB$ induces the following functions:
    \begin{itemize}
    \item $\lTYP \defeq \fun{\lE}{\{\lR, \lW, \lF\}}$ --- assigns a type to every event;
    \item $\lLOC \defeq \pfun{\lE}{Loc}$ --- returns the location of event (when applicable);
    \item $\lVALR \defeq \pfun{\lE}{Val}$ --- returns the read value of event (when applicable);
    \item $\lVALW \defeq \pfun{\lE}{Val}$ --- returns the written value of event (when applicable).
    \end{itemize}
    We assume that $\forall{e} \in \lEi. \; \lLAB(e) = \wlab{x}{0}$.
  \item A function $\lMOD = G.\lE \rightarrow \Mod$ is a memory order assignment.
    Elements of the set $\Mod \defeq \{ \rlx, \rel, \acq, \acqrel, \sco \}$ are
    called \emph{memory order} parameters.
    Additionally, $\lMOD$ satisfies the following constraints:
    \begin{itemize}
    \item $e \in \lR \Rightarrow \lMOD(e) \in \{ \rlx, \acq, \sco \}$,
    \item $e \in \lW \Rightarrow \lMOD(e) \in \{ \rlx, \rel, \sco \}$,
    \item $e \in \lF \Rightarrow \lMOD(e) \in \{ \rlx, \acqrel, \sco \}$.
    \end{itemize}
  \item $\lPO \subseteq \lE \times \lE$ --- \emph{program order}, which is a strict partial order.
    It orders all initialization events before all other events,
    that is $\lEi \times (\lE \setminus \lEi) \subseteq \lPO$.
    Moreover, $\lPO$ induces the \emph{conflict relation} $\lCF$.
    \begin{itemize}
      \item  $\lCF \defeq \lPO^{-1};[E];\lPO \setminus \lPO^{=}$.
    \end{itemize}
  \item $\lRMW \subseteq \lR;(\lPO_{imm} \cap =_{loc});\lW \times \lE$ ---
    \emph{read-modify-write pairs}.
  \item $\lRF \subseteq \lW; =_{loc}; \lR$ --- \emph{reads-from} relation, which the following
    hold for:
    \begin{itemize}
    \item $\forall{\tup{a, b}} \in \lRF. \; \lVALW(a) = \lVALR(b)$;
    \item $\forall{a_1, a_2, b}. \; \tup{a_1, b} \in \lRF \wedge \tup{a_2, b} \in \lRF \Rightarrow a_1 = a_2.$
    \end{itemize}
  \item $\lEW \subseteq \lW \times \lW$ --- is the \emph{equal write} relation, 
    which is an irreflexive, symmetric and transitive relation between conflicting writes on the
    same location writing same values 
    (that is $\lEW(e_1, e_2) \Rightarrow \lVALW(e_1) = \lVALW(e_2)$).
    Given $\lRF$, $\lEW$ and $\lCF$ we will also define a derived \emph{justified from} relation:
    \begin{itemize}
      \item $\lJF \defeq \lEW^?;\lRF \setminus \lCF$
    \end{itemize}
  \item $\lMO \subseteq \lW \times \lW$ --- \emph{modification order}, which is a strict partial order.
  \end{itemize}
\end{definition}

\begin{definition}
  Using the primitive relations of an event structure $\lES$ we define following derived relations:
  \begin{itemize}
    \item $\lES.\lRB \defeq (\lES.\lJF^{-1};\lES.\lMO) \setminus \lES.\lCF^?$ --- 
      \emph{reads before};
    \item $\lES.\lECO \defeq (\lES.\lJF \cup \lES.\lRB \cup \lES.\lMO)$ --- 
      \emph{extended coherence order}
  \end{itemize}
\end{definition}

\begin{definition}
  For an event structure $\lES$ the set of its visible events $\lVIS(\lES)$ defined as follow:
  \begin{itemize}
    \item $\lVIS(\lES) \defeq 
      \{ e \in \lES ~|~ \lES.\lW;(\lES.\lCF \cap (\lES.\lPO \cup \lES.\lRF)^+);[e] \subseteq 
         \lES.\lEW;\lES.\lPO^=
      \}$
  \end{itemize}
\end{definition}

\begin{definition}
  Given an event structure $\lES$ we will a call a tuple $\tup{\lE, \lPO, \lRMW, \lRF, \lMO}$ 
  its \emph{execution} $\lX$ if the following conditions hold:
  \begin{itemize}
    \item $\lX.\lE \subseteq \lVIS(\lES)$
    \item $[\lX.\lE];\lES.\lCF;[\lX.\lE] = \emptyset$
    \item $\lES.\lPO;[\lX.\lE] \subseteq \lX.\lE \times \lX.\lE$
    \item $\forall{\tup{e, e'}} \in (\lX.\lE;\lES.\lPO;[\lES.\lE \setminus \lX.\lE]) ~.~ 
      \exists{e''} \in \lX.\lE ~.~ \tup{e', e''} \in \lES.\lCF$
    \item $\lX.\lPO = \lES.\lPO \cap \lX.\lE \times \lX.\lE$
    \item $\lX.\lRMW = \lES.\lRMW \cap \lX.\lE \times \lX.\lE$
    \item $\lX.\lRF = \lES.\lJF \cap \lX.\lE \times \lX.\lE$
    \item $\lX.\lMO = \lES.\lMO \cap \lX.\lE \times \lX.\lE$
  \end{itemize}
\end{definition}

\begin{definition}
  A function $O_{\lX} :: \fun{\Loc}{\Val}$ is an \emph{outcome} 
  of an event structure's $\lES$ execution $\lX$
  if for every $x \in \Loc$ either $O_{\lX}(x) = \lES.\lVALW(w)$ 
  for some $\lX.\lMO$-maximal write event $w$, 
  or $O_{\lG} = 0$ and $\lX.\lW_x = \emptyset$.
\end{definition}

\section{Intermediate Memory Model}

\begin{definition}
  An \imm execution graph $\lG$ is a tuple \\
  $\tup{\lE, \lLAB, \lMOD, \lRMW, \lRF, \lMO, \lDATA, \lADDR, \lCTRL, \lCASDEP}$ where:
  \begin{itemize}
    \item $\lE$ is a set of events. 
      Events in \imm have different representation comparing to the event structure model. 
      That is, an event is either:
      \begin{itemize}
        \item an \emph{initialization} event $\tup{\lINIT~x}$ where $x \in \Loc$;
        \item a \emph{non-initialization} event $\tup{i, n}$ where $i \in \Tid$ and $n \in \Q$.
      \end{itemize}
      Given this representation one could restore some piece of the event structure notation, 
      in particular:
      \begin{itemize}
        \item $\lEi \defeq \{e \in E ~|~ \exists{x} \in \Loc ~.~ e = \tup{\lINIT~x}\}$;
        \item $\lTID :: \fun{\lE}{\Tid}$ s.t. \\
          $\forall{e} ~.~ (e = \tup{\lINIT~x} \implies \lTID(e) = 0) \wedge 
           (e = \tup{i, n} \Rightarrow \lTID(e) = i)$;
        \item $\lPO \subseteq \lE \times \lE$ s.t. \\
          $\tup{e_1, e_2} \in \lPO \iff (e_1 \in \lEi \wedge e_2 \not\in \lEi) \vee 
           (e_1 \not\in \lEi \wedge e_2 \not\in \lEi \wedge \lTID(e_1) = \lTID(e_2) \wedge
            \lSN(e_1) < \lSN(e_2))$, 
          where $\lSN(\tup{i, n}) = n$.
      \end{itemize}
      \item $\lLAB, \lMOD, \lRMW, \lRF, \lMO$ have the same definitions
        as in the event structure model.
      \item $\lDATA \subseteq \lR \times \lW$ --- data dependency;
      \item $\lADDR \subseteq \lR \times (\lR \cup \lW)$ --- address dependency;
      \item $\lCTRL \supseteq \lCTRL; \lPO$ --- control dependency;
      \item $\lCASDEP \subseteq [R];\lPO;[R^{\Ex}]$ --- CAS dependency;
  \end{itemize}
\end{definition}

\begin{definition}
  A function $O_{\lG} :: \fun{\Loc}{\Val}$ is an \emph{outcome} of an \imm execution graph $\lG$
  if for every $x \in \Loc$ either $O_{\lG}(x) = \lG.\lVALW(w)$ 
  for some $\lG.\lMO$-maximal write event $w$, 
  or $O_{\lG} = 0$ and $\lG.\lW_x = \emptyset$.
\end{definition}

\section{Compilation correctness}

\begin{theorem}
  For every consistent \imm execution graph $G$
  there exists a consistent event structure $\lES$ and its consistent execution $\lX$
  such that an outcome of execution graph $O_{\lG}$ matches
  an outcome of events structure's execution $O_{\lX}$, 
  that is $O_{\lG} = O_{\lX}$.
\end{theorem}

\subsection{Simulation relation}

\begin{definition}
  Simulation relation $\lSIM(\lES, \lG, \tup{\lC, \lI})$ that binds an 
  event structure $\lES$, an \imm execution graph $\lG$ and 
  a traversal configuration $\tup{\lC, \lI}$,
  holds if there exists a injecive mapping between events $\lM :: \fun{\lG.\lE}{\lES.\lE}$
  and a program counter $\lPC :: \fun{\Tid}{\lES.\lE}$
  such that the following conditions are met:
  \begin{itemize}
    \item $\forall{e \in \lC \cup \lI} ~.~ \lG.\lLAB(e) = \lES.\lLAB(\lM(e))$;
    \item $\forall{e \in \lC \cup \lI} ~.~ \lM(e) \in \lVIS(\lES)$;
    \item $\forall{i \in \Tid} ~.~ \lDOM(\lES.\lPO^?;[\lPC(i)]) = \lM(\lC) \cap \lES.\lE_i$;
    \item $\forall{\tup{w, r} \in \lG.\lRF;[\lC] ~.~ \tup{\lM(w), \lM(r)} \in \lES.\lJF}$;
    \item $\forall{\tup{w, w'} \in [\lI];\lG.\lMO;[\lI] ~.~ \tup{\lM(w), \lM(w')} \in \lES.\lMO}$;
  \end{itemize}
\end{definition}

\setmonofont[Mapping=tex-text]{CMU Typewriter Text}
\bibliographystyle{acm}
\bibliography{main.bib}

\end{document}
