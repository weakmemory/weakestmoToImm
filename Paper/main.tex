\documentclass[12pt]{article}
\usepackage{fontspec}
\usepackage{xcolor}
\usepackage{listings}
\usepackage[justification=centering]{caption}
\usepackage{subcaption}
\usepackage{cite}
\usepackage{amssymb}
\usepackage{amsmath}
\usepackage{amsthm}
\usepackage{mathtools}
\usepackage{bussproofs}
\usepackage{tikz}
\usepackage{float}
\usepackage{graphicx}
\usepackage{algorithm}
\usepackage[noend]{algpseudocode}
\usepackage{array}
\usepackage{stmaryrd}
\usepackage{pifont}
\usepackage{enumitem}
\usepackage{extarrows}

\usepackage{hyperref}
\usepackage[capitalise]{cleveref}
%% \usepackage{chngcntr}

\newcommand{\rulehskip}{\hskip 1.5em}
\newcommand{\rulevspace}{\vspace{1em}}

\mathchardef\mhyphen="2D

\theoremstyle{definition}

\newtheorem{definition}{Definition}
\newtheorem{theorem}{Theorem}
\newtheorem{lemma}{Lemma}
\newtheorem{prop}{Proposition}

%% \counterwithin{lemma}{section}

\newcommand{\textdef}[1]{\textit{#1}}

\newcommand{\imm}{{\textrm IMM}~}

% inline code 
\newcommand{\code}[1]{\texttt{#1}}

% tuple with angle brackets
\newcommand{\tup}[1]{\langle #1 \rangle}

% semantics brackets
\newcommand{\sem}[1]{\llbracket #1 \rrbracket}

% equality by definition
\newcommand{\defeq}{\triangleq}

% function arrow
\newcommand{\fun}{\rightarrow}

% partial function arrow
\newcommand{\pfun}{\rightharpoonup}

% some math sets
\newcommand{\N}{{\mathbb{N}}}
\newcommand{\Q}{{\mathbb{Q}}}

% domain/codomain notation
\newcommand{\dom}[1]{\textit{dom}{({#1})}}
\newcommand{\codom}[1]{\textit{codom}{({#1})}}

\newcommand{\lca}{\textit{lca}}

% some logical notation
%\newcommand{\implies}{{\Rightarrow}}
%\newcommand{\iff}{{\Leftrightarrow}}

% check-mark and cross-mark
\newcommand{\cmark}{\text{\color{green}\ding{51}}}
\newcommand{\xmark}{\text{\color{red}\ding{55}}}

% event structure
\newcommand{\lES}{S}
% event structure's execution

% execution graph
\newcommand{\lG}{G}
% covered/issued events
\newcommand{\lC}{C}
\newcommand{\lI}{I}
% mapping between events
\newcommand{\lM}{m}
\newcommand{\lMC}{m_{c}}
\newcommand{\lMI}{m_{i}}

% event set
\newcommand{\lE}{{\mathtt{E}}}
\newcommand{\lEi}[1]{\lE_{#1}}
\newcommand{\lEo}{\lEi{0}}
\newcommand{\lEs}{\widehat{{\mathtt{E}}}}
\newcommand{\lEsi}[1]{\lEs_{#1}}
\newcommand{\lEso}{\lEsi{0}}
\newcommand{\lR}{{\mathtt{R}}}
\newcommand{\lW}{{\mathtt{W}}}
\newcommand{\lF}{{\mathtt{F}}}
\newcommand{\lX}{{\mathtt{X}}}

\newcommand{\lEsqeq}[1]{{\lE^{\sqsupseteq#1}}}
\newcommand{\lRsqeq}[1]{{\lR^{\sqsupseteq#1}}}
\newcommand{\lWsqeq}[1]{{\lW^{\sqsupseteq#1}}}
\newcommand{\lFsqeq}[1]{{\lF^{\sqsupseteq#1}}}

% labeling of events
\newcommand{\Tid}{\mathsf{Tid}}
\newcommand{\Loc}{\mathsf{Loc}}
\newcommand{\Val}{\mathsf{Val}}
\newcommand{\Bool}{\mathsf{Bool}}
\newcommand{\Lab}{\mathsf{Lab}}
\newcommand{\Mod}{\mathsf{Mod}}
\newcommand{\Cont}{\kappa}
\newcommand{\Scp}{\mathsf{Scope}}
\newcommand{\Path}{\pi}
\newcommand{\Prog}{\mathsf{prog}}
\newcommand{\Execs}[1]{\Psi_{#1}}

\newcommand{\lLAB}{{\mathtt{lab}}}
\newcommand{\lTID}{{\mathtt{tid}}}
\newcommand{\lTYP}{{\mathtt{typ}}}
\newcommand{\lLOC}{{\mathtt{loc}}}
\newcommand{\lMOD}{{\mathtt{mod}}}
\newcommand{\lSCP}{{\mathtt{scp}}}
\newcommand{\lCONT}{{\mathtt{K}}}
\newcommand{\lVALR}{{\mathtt{val_r}}}
\newcommand{\lVALW}{{\mathtt{val_w}}}
\newcommand{\lINIT}{{\mathtt{init}}}
\newcommand{\lSN}{{\mathtt{sn}}}
\newcommand{\lPC}{{\mathtt{pc}}}
\newcommand{\lPREV}{{\mathtt{prev}}}

\newcommand{\lMASK}{{\mathtt{Mask}}}
\newcommand{\lVIS}{{\mathtt{Vis}}}

% simulation relation 
\newcommand{\simR}{\mathcal{I}}
\newcommand{\simRb}{\mathcal{I}_{base}}
\newcommand{\simRw}{\mathcal{I}_{weak}}

\newcommand{\nextset}{{\mathtt{Next}}}
\newcommand{\coverable}{{\mathtt{Coverable}}}
\newcommand{\issuable}{{\mathtt{Issuable}}}
\newcommand{\front}{\phi}
\newcommand{\forward}{{\mathtt{Forward}}}
\newcommand{\cert}{{\mathtt{cert}}}
\newcommand{\certify}{{\mathtt{Certify}}}

%\newcommand{\lCERTIFY}{{\mathtt{Certify}}}

\newcommand{\trstep}{\longrightarrow}
\newcommand{\esbstep}[1]{\xLongrightarrow{#1}}
\newcommand{\esstep}[1]{\xlongrightarrow{#1}}
\newcommand{\esstepstar}{\longrightarrow^*}

\newcommand{\ESle}{\preceq}
\newcommand{\ESbasicstep}[1]{\xLongrightarrow{#1}}
\newcommand{\ESstep}[1]{\xlongrightarrow{#1}}
\newcommand{\ESstepstar}{\longrightarrow^*}

\newcommand{\rlab}[3]{\lR^{#1}({#2},{#3})}
\newcommand{\wlab}[3]{\lW^{#1}({#2},{#3})}
\newcommand{\flab}[1]{\lF^{#1}}

\newcommand{\Ex}{{\mathtt{ex}}}
\newcommand{\NotEx}{{\mathtt{not \mhyphen ex}}}
\newcommand{\rlabExExpl}[4]{\lR_{#4}^{#1}({#2},{#3})}
\newcommand{\rlabEx}[3]{\rlabExExpl{#1}{#2}{#3}{\Ex}}

%% \newcommand{\rlabS}[3]{\lR^{#3}({#1},{#2})}
%% \newcommand{\wlabS}[3]{\lW^{#3}({#1},{#2})}
%% \newcommand{\flabS}[1]{\lF^{#1}}

% some predefined relations
\colorlet{colorPO}{gray!60!black}
\colorlet{colorPPO}{magenta}
\colorlet{colorRF}{green!60!black}
\colorlet{colorJF}{green!40!black}
\colorlet{colorRMW}{olive!70!black}
\colorlet{colorCF}{red!60!black}
\colorlet{colorEW}{brown}
\colorlet{colorMO}{orange!60!black}
\colorlet{colorRB}{purple}
\colorlet{colorECO}{red!80!black}
\colorlet{colorRSEQ}{cyan}
\colorlet{colorRELP}{cyan}
\colorlet{colorSW}{blue!40!black}
\colorlet{colorHB}{blue}
\colorlet{colorSCB}{violet}
\colorlet{colorPSC}{violet}
\colorlet{colorFSC}{violet}
\colorlet{colorSC}{violet}
\colorlet{colorWGR}{magenta!40!black}
\colorlet{colorDEV}{magenta}
\colorlet{colorINCL}{magenta!70!black}
\colorlet{colorOBS}{green!40!black}
\colorlet{colorCA}{teal}
\colorlet{colorDEPS}{violet!60!black}
\colorlet{colorDETOUR}{teal}
\colorlet{colorBOB}{gray!80!black}

\newcommand{\lPO}{{\color{colorPO}\mathtt{po}}}
\newcommand{\lPPO}{{\color{colorPPO}\mathtt{ppo}}}
\newcommand{\lRF}{{\color{colorRF}\mathtt{rf}}}
\newcommand{\lJF}{{\color{colorRF}\mathtt{jf}}}
\newcommand{\lRMW}{{\color{colorRMW}\mathtt{rmw}}}
\newcommand{\lCF}{{\color{colorCF}\mathtt{cf}}}
\newcommand{\lEW}{{\color{colorEW}\mathtt{ew}}}
\newcommand{\lMO}{{\color{colorMO}\mathtt{mo}}}
\newcommand{\lRB}{{\color{colorRB}\mathtt{rb}}}
\newcommand{\lECO}{{\color{colorECO}\mathtt{eco}}}

\newcommand{\lRSEQ}{{\color{colorRSEQ}\mathtt{rseq}}}
\newcommand{\lRELP}{{\color{colorRELP}\mathtt{relp}}}
\newcommand{\lSW}{{\color{colorSW}\mathtt{sw}}}
\newcommand{\lHB}{{\color{colorHB}\mathtt{hb}}}

\newcommand{\ocl}{{\textrm OpenCL}}
\newcommand{\ptx}{{\textrm PTX}}
\newcommand{\lSWocl}{{\color{colorSW}\mathtt{sw_\ocl}}}
\newcommand{\lSWptx}{{\color{colorSW}\mathtt{sw_\ptx}}}

\newcommand{\lSCB}{{\color{colorSCB} \mathtt{scb}}}
\newcommand{\lPSCB}{\lPSC_{\rm base}}
\newcommand{\lPSCF}{\lPSC_\lF}
\newcommand{\lFSC}{{\color{colorFSC}\mathtt{fsc}}}
\newcommand{\lPSC}{{\color{colorPSC}\mathtt{psc}}}
\newcommand{\lSC}{{\color{colorSC}\mathtt{sc}}}

\newcommand{\lWGR}{{\color{colorWGR}\mathtt{wgr}}}
\newcommand{\lWGRi}[1]{{\color{colorWGR}\mathtt{wgr}_{#1}}}
\newcommand{\lDEV}{{\color{colorDEV}\mathtt{dev}}}
\newcommand{\lINCL}{{\color{colorINCL}\mathtt{incl}}}
\newcommand{\lHBINCL}{\lHB_{\lINCL}}

\newcommand{\lOBS}{{\color{colorOBS}\mathtt{obs}}}
\newcommand{\lCA}{{\color{colorCA}\mathtt{ca}}}
\newcommand{\lCAB}{\lCA_{\rm base}}

\newcommand{\lCTRL}{{{\color{colorDEPS}\mathtt{ctrl}}}}
\newcommand{\lDATA}{{{\color{colorDEPS}\mathtt{data}}}}
\newcommand{\lADDR}{{{\color{colorDEPS}\mathtt{addr}}}}
\newcommand{\lCASDEP}{{{\color{colorDEPS}\mathtt{casdep}}}}
\newcommand{\lDEPS}{{\color{colorDEPS}\mathtt{dep}}}

\newcommand{\lBOB}{{\color{colorBOB}\mathtt{bob}}}
\newcommand{\lDETOUR}{{\color{colorDETOUR}\mathtt{detour}}}

\newcommand{\lmakeE}[1]{#1\mathtt{e}}
\newcommand{\lRFE}{\lmakeE{\lRF}}
\newcommand{\lCOE}{\lmakeE{\lCO}}
\newcommand{\lRBE}{\lmakeE{\lRB}}
\newcommand{\lMOE}{\lmakeE{\lMO}}

\newcommand{\lmakeI}[1]{#1\mathtt{i}}
\newcommand{\lRFI}{\lmakeI{\lRF}}
\newcommand{\lCOI}{\lmakeI{\lCO}}
\newcommand{\lFRI}{\lmakeI{\lFR}}

\newcommand{\lmakeLoc}[1]{#1_{|loc}}

%% memory orders
\newcommand{\na}{\mathtt{na}}
\newcommand{\pln}{\mathtt{pln}}
\newcommand{\rlx}{\mathtt{rlx}}
\newcommand{\rel}{{\mathtt{rel}}}
\newcommand{\acq}{{\mathtt{acq}}}
\newcommand{\con}{{\mathtt{con}}}
\newcommand{\acqrel}{{\mathtt{acqrel}}}
\newcommand{\relAcq}{\acqrel}
\newcommand{\sco}{{\mathtt{sc}}}

%% scopes
\newcommand{\wgr}{\mathtt{wgr}}
\newcommand{\dev}{\mathtt{dev}}
\newcommand{\sys}{\mathtt{sys}}

%% tikz stuff

\newcommand{\event}[3]{#1#2#3}
\tikzset{
   every path/.style={>=stealth},
   po/.style={->,color=colorPO,,shorten >=-0.5mm,shorten <=-0.5mm},
   rf/.style={->,color=colorRF,dashed,,shorten >=-0.5mm,shorten <=-0.5mm},
   rb/.style={->,color=colorRB,thick,shorten >=-0.5mm,shorten <=-0.5mm},
   mo/.style={->,color=colorMO,dotted,thick,shorten >=-0.5mm,shorten <=-0.5mm},
   sw/.style={->,color=colorSW,dashed,thick,shorten >=-0.5mm,shorten <=-0.5mm},
   obs/.style={->,color=colorOBS,dashed,,shorten >=-0.5mm,shorten <=-0.5mm},
   no/.style={->,dotted,thick,shorten >=-0.5mm,shorten <=-0.5mm},
   deps/.style={->,color=colorDEPS,dotted,thick,shorten >=-0.5mm,shorten <=-0.5mm},
   wgr/.style={fill=colorWGR, opacity=0.1}
}

%% parallel threads listings

\newcommand{\inarr}[1]{\begin{array}{@{}l@{}}#1\end{array}}
\newcommand{\inarrII}[2]{\begin{array}{@{}l@{~~}||@{~~}l@{}}\inarr{#1}&\inarr{#2}\end{array}}
\newcommand{\inarrIII}[3]{\begin{array}{@{}l@{~~}||@{~~}l@{~~}||@{~~}l@{}}\inarr{#1}&\inarr{#2}&\inarr{#3}\end{array}}
\newcommand{\inarrIV}[4]{\begin{array}{@{}l@{~~}||@{~~}l@{~~}||@{~~}l@{~~}||@{~~}l@{}}\inarr{#1}&\inarr{#2}&\inarr{#3}&\inarr{#4}\end{array}}
\newcommand{\inarrV}[5]{\begin{array}{@{}l@{~~}||@{~~}l@{~~}||@{~~}l@{~~}||@{~~}l@{~~}||@{~~}l@{}}\inarr{#1}&\inarr{#2}&\inarr{#3}&\inarr{#4}&\inarr{#5}\end{array}}

%% instructions for listings

\newcommand{\valueRead}[1]{{// #1}}

\newcommand{\fence}{\texttt{fence}}
\newcommand{\readInstS}[3]{#2 \;:=^{#1}\;[#3]}
\newcommand{\writeInstS}[3]{[#2] \;:=^{#1}\;#3}
\newcommand{\fenceInstS}[1]{\fence^{#1}}

%% axiom labels

\newcounter{mylabelcounter}

\makeatletter
\newcommand{\labelAxiom}[2]{%
\hfill{\normalfont\textsc{(#1)}}\refstepcounter{mylabelcounter}
\immediate\write\@auxout{%
  \string\newlabel{#2}{{\unexpanded{\normalfont\textsc{#1}}}{\thepage}{{\unexpanded{\normalfont\textsc{#1}}}}{mylabelcounter.\number\value{mylabelcounter}}{}}
}%
}
\makeatother

%% warning

\colorlet{colorWARNING}{yellow!90!black}

\newcommand{\warning}[1]{{\color{colorWARNING}\texttt{WARNING}}: #1}
\newcommand{\app}[1]{{\color{blue}\textbf{ANTON: #1}}}
\newcommand{\note}[1]{{\color{cyan}\textbf{EVG: #1}}}
\newcommand{\todo}[1]{{\color{red}\textbf{TODO: #1}}}


\begin{document}

\begin{center}
{\center \LARGE Compiling Event Structures to the Intermediate Memory Model }
\end{center}

%% \begin{definition}
%%   \label{def:exec}
  
%%   An execution graph $X$ is a tuple $\tup{\lE, \lLAB, \lRF, \lMO}$ where:
%%   \begin{itemize}

%%   \item $\lE$ is a set of events. 
%%     Execution events have different representation comparing to the event structure events. 
%%     That is, an event is either:
%%     \begin{itemize}
%%       \item an \emph{initialization} event $\tup{\lINIT~x}$ where $x \in \Loc$;
%%       \item a \emph{non-initialization} event $\tup{i, n}$ where $i \in \Tid$ and $n \in \Q$.
%%     \end{itemize}
%%     Given this representation one could derive the following notions:
%%     \begin{itemize}

%%       \item $\lEo \defeq \{e \in E ~|~ \exists{x} \in \Loc ~.~ e = \tup{\lINIT~x}\}$;

%%       \item $\lTID : \lX \fun \Tid$ s.t. \\
%%         \begin{equation*}
%%           \begin{split}
%%             \forall{e} ~.~
%%             & (\exists{x \in \Loc} ~.~ e = \tup{\lINIT~x} \Rightarrow \lTID(e) = 0) \wedge \\ \wedge
%%             & (\exists{i \in \Tid, n \in \N} ~.~ e = \tup{i, n} \Rightarrow \lTID(e) = i)
%%           \end{split}
%%         \end{equation*}
        
%%       \item $\lPO \subseteq \lE \times \lE$ s.t. \\
%%         \begin{equation*}
%%           \begin{split}
%%             & \tup{e_1, e_2} \in \lPO \Leftrightarrow \\
%%             & \Leftrightarrow (e_1 \in \lEo \wedge e_2 \not\in \lEo) \vee 
%%               (e_1, e_2 \not\in \lEo \wedge \lTID(e_1) = \lTID(e_2) \wedge
%%               \lSN(e_1) < \lSN(e_2))
%%           \end{split}
%%         \end{equation*}
%%         where $\lSN(\tup{i, n}) = n$.

%%         Note that $\lPO$ totally orders all events within a thread.
%%     \end{itemize}

%%   \end{itemize}
%% \end{definition}

%% \begin{definition}
%%   Given an event structure $S$ we will a call an execution graph $X$
%%   extracted from $S$, denoted as $S \rhd X$,
%%   if there exists a function $s : X.\lE \fun S.\lE$, called \emph{source event mapping},
%%   such that the following conditions are met.
%%   \begin{itemize}
%%     \item $s(X.\lE) \subseteq \lVIS(S)$;
%%     \item $[s(X.\lE)];S.\lCF;[s(X.\lE)] = \emptyset$;
%%     \item $S.\lPO;[s(X.\lE)] \subseteq s(X.\lE) \times s(X.\lE)$;
%%     \item $\forall{\tup{e, e'}} \in [s(X.\lE)];S.\lPO;[S.\lE \setminus s(X.\lE)] ~.~ 
%%       \exists{e''} \in s(X.\lE) ~.~ \tup{e', e''} \in S.\lCF$;
%%     \item $X.\lPO = [s(X.\lE)];S.\lPO;[s(X.\lE)]$;
%%     \item $X.\lRMW = [s(X.\lE)];S.\lRMW;[s(X.\lE)]$;
%%     \item $X.\lRF = [s(X.\lE)];S.\lRF;[s(X.\lE)]$;
%%     \item $X.\lMO = [s(X.\lE)];S.\lMO;[s(X.\lE)]$;
%%   \end{itemize}
%% \end{definition}

%% \begin{definition}
%%   We will call an execution graph $X$ \emph{consistent} if the following properties hold:
%%   \begin{itemize}
%%     \item $X.\lMO^= = [X.\lW];=_{\lLOC};[X.\lW]$
%%     \item $X.\lRMW \cap (X.\lRB;X.\lMO) = \emptyset$
%%     \item $X.\lHB;X.\lECO^?$ is irreflexive.
%%   \end{itemize}
%% \end{definition}

%% \begin{definition}
%%   A function $O : \Loc \fun \Val$ is an \emph{outcome} 
%%   of a consistent execution graph $X$
%%   if for every $x \in \Loc$ either $O(x) = X.\lVALW(w)$ 
%%   for some $X.\lMO$-maximal write event $w$, 
%%   or $O(x) = 0$ and $X.\lW_x = \emptyset$.
%% \end{definition}

\section{Event Structure Model}

\begin{definition}
  \label{def:es}
  
  An \emph{event structure} $S$ is a tuple
  $\tup{\lEs, \lJF, \lEW, \lMO}$ where:
  \begin{itemize}
    \item $\lEs$ is a set of \emph{s-events}.
    An s-event $\hat{e} \in \lEs$ is a tuple $\tup{i, \Path}$
    where $i \in \Tid$ and $\Path = (l_1, \dots, l_n)$ is a finite sequence of labels.
    Giving $\lEs$ we will also derive the following definitions:
    \item $\lTID : \lE \fun \Tid$ --- function that assigns a thread id to every event.
      \begin{equation*}
        \lTID(\tup{i, \Path}) \defeq i
      \end{equation*}
      Given thread id $i \in \Tid$ we will denote by $\lEi{i}$ the set of all events belonging 
      to $i^{th}$ thread, that is $\lEi{i} \defeq \{e \in \lE ~|~ \lTID(e) = i\}$.
      We assume that if $e \in \lEo$ then $\lTID(e) = 0$.
    
  \item $\lE \subseteq \N$ --- finite set of s-events. 
  \item $\lEo \subseteq \lE$ --- initialization events.
  \item $\lTID \defeq \lE \fun \Tid$ --- function that assigns a thread id to every event.
    Given thread id $i \in \Tid$ we will denote by $\lEi{i}$ the set of all events belonging 
    to $i^{th}$ thread, that is $\lEi{i} \defeq \{e \in \lE ~|~ \lTID(e) = i\}$.
    We assume that if $e \in \lEo$ then $\lTID(e) = 0$.
  \item $\lLAB \defeq \lE \fun \Lab$ --- function that assigns a label to every event.
    Labels are of one of the following forms:
    \begin{itemize}
      \item $\rlabExExpl{o}{x}{v}{\mathtt{f}}$ --- a read, where
        $x \in \Loc$, $v \in \Val$, $o \in \Mod$ 
        and $\mathtt{f} \in \{\Ex, \NotEx\}$ is an exclusive flag 
        (we will denote exclusive reads as $\rlabEx{o}{x}{v}$,
        for non-exclusive reads we will omit the flag);
      \item $\wlab{o}{x}{v}$ --- a write, where $x \in \Loc$, $v \in \Val$, $o \in \Mod$;
      \item $\flab{o}$ --- a fence, where $o \in Mod$.
    \end{itemize}
    We assume that $\forall{e} \in \lEo. \; \lLAB(e) = \wlab{\rlx}{x}{0}$.
    $\lLAB$ induces the following functions:
    \begin{itemize}
      \item $\lTYP \defeq \lE \fun \{\lR, \lW, \lF\}$ --- assigns a type to every event;
      \item $\lLOC \defeq \lE \pfun \Loc $ --- returns the location of event (when applicable);
      \item $\lVALR \defeq \lE \pfun \Val$ --- returns the read value of event (when applicable);
      \item $\lVALW \defeq \lE \pfun \Val$ --- returns the written value of event
        (when applicable);
      \item $\lMOD = G.\lE \rightarrow \Mod$ --- returns the associated memory order parameter.
        Additionally, $\lMOD$ satisfies the following constraints:
        \begin{itemize}
        \item $e \in \lR \Rightarrow \lMOD(e) \in \{ \rlx, \acq, \sco \}$;
        \item $e \in \lW \Rightarrow \lMOD(e) \in \{ \rlx, \rel, \sco \}$;
        \item $e \in \lF \Rightarrow \lMOD(e) \in \{ \rlx, \acqrel, \sco \}$.
        \end{itemize}
    \end{itemize}
  \item $\lCONT$ --- a list of \emph{continuations}.
  \item $\lPO \subseteq \lE \times \lE$ --- \emph{program order},
    which is a strict partial order.
    It orders all initialization events before all other events,
    that is $\lEo \times (\lE \setminus \lEo) \subseteq \lPO$.
  \item $\lCF \subseteq [\lE];=_{\lTID};[\lE]$ --- \emph{conflict relation},
    which binds two events whenever they are issued from the same thread but
    in different execution branches.    
  \item $\lRMW \subseteq [\lR];(\lPO_{imm} \cap =_{\lLOC});[\lW]$ ---
    \emph{read-modify-write pairs}.
  \item $\lJF \subseteq [\lW];=_{\lLOC};[\lR]$ --- \emph{reads-from} relation, which the following
    hold for:
    \begin{itemize}
    \item $\forall{\tup{a, b}} \in \lJF. \; \lVALW(a) = \lVALR(b)$;
    \item $\forall{a_1, a_2, b}. \;
      \tup{a_1, b} \in \lJF \wedge \tup{a_2, b} \in \lJF \Rightarrow a_1 = a_2.$
    \end{itemize}
  \item $\lEW \subseteq \lW \times \lW$ --- is the \emph{equal write} relation, 
    which is an irreflexive, symmetric and transitive relation between conflicting writes on the
    same location writing same values 
    (that is $\lEW(e_1, e_2) \Rightarrow \lVALW(e_1) = \lVALW(e_2)$).
    Given $\lJF$, $\lEW$ and $\lCF$ we will also define a derived \emph{read from} relation:
    \begin{itemize}
      \item $\lRF \defeq \lEW^?;\lJF \setminus \lCF$
    \end{itemize}
  \item $\lMO \subseteq \lW \times \lW$ --- \emph{modification order},
    which is a strict partial order.
  \end{itemize}
\end{definition}

\begin{definition}
  Using the primitive relations of an event structure $\lES$,
  we define following derived relations:
  \begin{itemize}
    \item $\lES.\lRB \defeq (\lES.\lRF^{-1};\lES.\lMO) \setminus \lES.\lCF^?$ --- 
      \emph{reads before};
    \item $\lES.\lECO \defeq (\lES.\lRF \cup \lES.\lRB \cup \lES.\lMO)^+$ --- 
      \emph{extended coherence order}.
  \end{itemize}
\end{definition}

\begin{definition}
  Giving an event structure $\lES$ we define the \emph{happens-before} relation $\lHB$
  using auxililarly relations: 
  $\lRSEQ$ --- release sequence,
  $\lRELP$ --- release prefix and
  $\lSW$ --- synchronize-with.
  \begin{itemize}
  \item $\lRSEQ \defeq [\lWsqeq{rlx}];(\lmakeLoc{\lPO};[\lWsqeq{rlx}] \cup 
    (\lmakeLoc{\lPO^?};[\lWsqeq{\rlx}];\lJF;\lRMW)^*)$;
  \item $\lRELP \defeq [\lEsqeq{rel}];([\lW] \cup [\lF];\lPO);\lRSEQ$;
  \item $\lSW \defeq \lRELP; (\lRFI \cup \lmakeLoc{\lPO^?};\lRFE); [\lR \cup \lPO;[\lF]);[\lEsqeq{\acq}]$;
  \item $\lHB \defeq (\lPO \cup \lSW)^+$.
  \end{itemize}
\end{definition}

%% \begin{definition}
%%   For an event $e$ from an event structure $\lES$
%%   the \emph{masking set} of $e$, denoted as $\lMASK(\lES, e)$,
%%   is a set of conflicting write events that $e$ depends on.
%%   This set is called masking, because its elements might make an event $e$ \emph{invisible}.
%%   \begin{itemize}
%%   \item $\lMASK(\lES, e) \defeq
%%     \{w \in \lES.\lW ~|~ \tup{w, e} \in \lES.\lCF \cap (\lES.\lPO \cup \lES.\lRF)^+ \}$
%%   \end{itemize}
%% \end{definition}

\begin{definition}
  For an event structure $S$ the set of its visible events $\lVIS(S)$ defined as follow:
  \begin{itemize}
    \item $\lVIS(S) \defeq 
      \{ e \in S ~|~ S.\lW;(S.\lCF \cap (S.\lPO \cup S.\lJF)^+);[e] \subseteq 
         S.\lEW;S.\lPO^=
      \}$,
  \end{itemize}
  %% or equivalently:
  %% \begin{itemize}
  %%   \item $\lVIS(\lES) \defeq 
  %%     \{ e \in \lES ~|~ \forall{w \in \lMASK(\lES, e)} ~.~ \tup{w, e} \in \lES.\lEW;\lES.\lPO^=\}$.
  %% \end{itemize}
\end{definition}

\begin{definition}
  Event structure $S$ is called \emph{consistent} if the following conditions hold:
  
  \begin{itemize}

    \item $S.\lJF \subseteq S.\lPO \cup (\lVIS(S) \times S.\lE)$;

    \item $S.\lJF \cap S.\lCF = \emptyset$;

    \item $(S.\lHB;S.\lJF^{-1}) \cap S.\lCF = \emptyset$;

    \item $S.\lHB;S.\lECO^?$ is irreflexive.
  \end{itemize}
\end{definition}

\section{Executions}

\begin{definition}
  \label{def:exec}
  
  An execution graph $X$ is a tuple $\tup{\lE, \lLAB, \lRMW, \lRF, \lMO}$ where:
  \begin{itemize}

  \item $\lE$ is a set of events. 
    Execution events have different representation comparing to the event structure events. 
    That is, an event is either:
    \begin{itemize}
      \item an \emph{initialization} event $\tup{\lINIT~x}$ where $x \in \Loc$;
      \item a \emph{non-initialization} event $\tup{i, n}$ where $i \in \Tid$ and $n \in \Q$.
    \end{itemize}
    Given this representation one could restore some piece of the event structure notation, 
    in particular:
    \begin{itemize}

      \item $\lEo \defeq \{e \in E ~|~ \exists{x} \in \Loc ~.~ e = \tup{\lINIT~x}\}$;

      \item $\lTID : \lX \fun \Tid$ s.t. \\
        \begin{equation*}
          \begin{split}
            \forall{e} ~.~
            & (\exists{x \in \Loc} ~.~ e = \tup{\lINIT~x} \Rightarrow \lTID(e) = 0) \wedge \\ \wedge
            & (\exists{i \in \Tid, n \in \N} ~.~ e = \tup{i, n} \Rightarrow \lTID(e) = i)
          \end{split}
        \end{equation*}
        
      \item $\lPO \subseteq \lE \times \lE$ s.t. \\
        \begin{equation*}
          \begin{split}
            & \tup{e_1, e_2} \in \lPO \Leftrightarrow \\
            & \Leftrightarrow (e_1 \in \lEo \wedge e_2 \not\in \lEo) \vee 
              (e_1, e_2 \not\in \lEo \wedge \lTID(e_1) = \lTID(e_2) \wedge
              \lSN(e_1) < \lSN(e_2))
          \end{split}
        \end{equation*}
        where $\lSN(\tup{i, n}) = n$.

        Note that in case of executions $\lPO$ is a total order.
    \end{itemize}
    
  \item $\lLAB, \lRMW, \lRF, \lMO$ have definitions similar to those given
    in the definition \ref{def:es} of event structure,
    except that they are defined on a set $X.\lE$.

  \end{itemize}
\end{definition}

\begin{definition}
  Given an event structure $S$ we will a call an execution graph $X$
  extracted from $S$, denoted as $S \rhd X$,
  if there exists a function $s : X.\lE \fun S.\lE$, called \emph{source event mapping},
  such that the following conditions are met.
  \begin{itemize}
    \item $s(X.\lE) \subseteq \lVIS(S)$;
    \item $[s(X.\lE)];S.\lCF;[s(X.\lE)] = \emptyset$;
    \item $S.\lPO;[s(X.\lE)] \subseteq s(X.\lE) \times s(X.\lE)$;
    \item $\forall{\tup{e, e'}} \in [s(X.\lE)];S.\lPO;[S.\lE \setminus s(X.\lE)] ~.~ 
      \exists{e''} \in s(X.\lE) ~.~ \tup{e', e''} \in S.\lCF$;
    \item $X.\lPO = [s(X.\lE)];S.\lPO;[s(X.\lE)]$;
    \item $X.\lRMW = [s(X.\lE)];S.\lRMW;[s(X.\lE)]$;
    \item $X.\lRF = [s(X.\lE)];S.\lRF;[s(X.\lE)]$;
    \item $X.\lMO = [s(X.\lE)];S.\lMO;[s(X.\lE)]$;
  \end{itemize}
\end{definition}

\begin{definition}
  We will call an execution graph $X$ \emph{consistent} if the following properties hold:
  \begin{itemize}
    \item $X.\lMO^= = [X.\lW];=_{\lLOC};[X.\lW]$
    \item $X.\lRMW \cap (X.\lRB;X.\lMO) = \emptyset$
    \item $X.\lHB;X.\lECO^?$ is irreflexive.
  \end{itemize}
\end{definition}

\begin{definition}
  A function $O : \Loc \fun \Val$ is an \emph{outcome} 
  of a consistent execution graph $X$
  if for every $x \in \Loc$ either $O(x) = X.\lVALW(w)$ 
  for some $X.\lMO$-maximal write event $w$, 
  or $O(x) = 0$ and $X.\lW_x = \emptyset$.
\end{definition}

\section{Intermediate Memory Model}

\begin{definition}
  An \imm execution graph $G$ is a tuple \\
  $\tup{X, \lDATA, \lADDR, \lCTRL, \lCASDEP}$ where:
  \begin{itemize}
    \item $X$ is a regular execution graph (from the definition \ref{def:exec}).
      We will omit the $X$ when refering to its components in the context of $G$
      (e.g. we will write $G.\lPO$ instead of $G.X.\lPO$).
    \item $\lDATA \subseteq \lR \times \lW$ --- data dependency.
    \item $\lADDR \subseteq \lR \times (\lR \cup \lW)$ --- address dependency.
    \item $\lCTRL \supseteq \lCTRL; \lPO$ --- control dependency.
    \item $\lCASDEP \subseteq [R];\lPO;[R^{\Ex}]$ --- CAS dependency.
  \end{itemize}

  We will also need several derived relations:
  \begin{itemize}
    \item $\lBOB \defeq
      \lPO;[\lW^{\rel}] \cup [R^{\acq}];\lPO \cup \lPO;[\lF] \cup [\lF];\lPO \cup [\lW^{\rel}];\lPO_{\lLOC};[\lW]$;
    \item $\lDEPS \defeq \lDATA \cup \lCTRL \cup \lADDR;\lPO^? \cup \lCASDEP \cup [\lR_{\Ex}];\lPO$;
    \item $\lPPO \defeq [\lR];(\lDEPS \cup \lRFI)^+;[\lW]$;
    \item $\lDETOUR \defeq (\lMOE;\lRFE) \cap \lPO$.
  \end{itemize}

\end{definition}

\begin{definition}
  An \imm execution graph $G$ is called \emph{\imm-consistent} if the following properties hold:
  \begin{itemize}
    \item $G.X$ is a consistent execution;
    \item $G.\lRFE \cup G.\lBOB \cup G.\lPPO \cup G.\lDETOUR$ is acyclic.
  \end{itemize}
\end{definition}

\section{Compilation correctness}

\begin{theorem}
  For every consistent \imm execution graph $G$
  there exists a consistent event structure $S$
  s.t. $S \rhd G.X$.
\end{theorem}

\subsection{Traversal}

\begin{definition}
  An event $e \in G.\lE$ is \emph{coverable} in \imm execution graph $G$ and $\tup{C, I}$,
  denoted $e \in \coverable(G, C, I)$, 
  if $\dom{G.\lPO;[e]} \subseteq C$ and either:
  \begin{itemize}
    \item $e \in G.\lW \cap I$;
    \item $e \in G.\lR \wedge \dom{G.\lRF;[e]} \subseteq I$;
    \item $e \in G.\lF^{\sqsubset\sco}$;
    \item $e \in G.\lF^{\sco} \wedge \dom{G.\lSC;[e]} \subseteq C$.
  \end{itemize}
\end{definition}

\begin{definition}
  A write event $w \in G.\lW$ is \emph{issuable} in \imm execution graph $G$ and $\tup{C, I}$,
  denoted $w \in \issuable(G, C, I)$, 
  if the following conditions are met:
  \begin{itemize}
    \item $\dom{([G.\lW^{rel}];\lmakeLoc{G.\lPO} \cup [G.\lF];G.\lPO);[w]} \subseteq C$
    \item $\dom{(G.\lDETOUR \cup G.\lRFE);G.\lPPO;[w]} \subseteq I$
    \item $\dom{(G.\lDETOUR \cup G.\lRFE);[G.R^{acq}];G.\lPO;[w]} \subseteq I$
  \end{itemize}
\end{definition}

\begin{definition}
  A traversal configuration of an \imm execution graph $G$ is a pair $\tup{C, I}$, 
  where $C \subseteq G.\lE$ is a set of \emph{covered} events 
  and $I \subseteq G.\lW$ is a set of \emph{issued} events,
  that additionally satisfies the following properties:
  \begin{itemize}
    \item $G.\lEo \subseteq C$;
    \item $C \cap G.\lW \subseteq I$;
    \item $C \subseteq \coverable(G, C, I)$;
    \item $I \subseteq \issuable(G, C, I)$;
  \end{itemize}
  Configuration is called \emph{initial} when $C = I = G.\lEo$.
\end{definition}

\begin{definition}
  For an \imm execution graph $G$ and a traversal configuration $\tup{C, I}$
  a set of \emph{next events} is a set defined as follows:
  \begin{itemize}
    \item $\nextset(G, C, I) \defeq \{e \in G.\lE ~|~ \dom{G.\lPO;[e]} \subseteq C \} \setminus C$
  \end{itemize}
\end{definition}

\begin{definition}
  A traversal step relation $G \vdash \tup{C, I} \trstep \tup{C', I'}$
  is defined according to rules on figure~\ref{fig:traversal-step}.
\end{definition}

\begin{figure}[thb]

\small
    
    \begin{center}
    \AxiomC{$e \in \nextset(G, C) \cap \coverable(G, C, I)$}
    \UnaryInfC{$
      G \vdash \tup{C, I} \rightarrow \tup{C \uplus \{e\}, I}
    $}
    \DisplayProof
    % 
    \rulehskip
    % 
    \AxiomC{$w \in \issuable(G, C, I) \setminus I$}
    \UnaryInfC{$
      G \vdash \tup{C, I} \rightarrow \tup{C, I \uplus \{w\}}
    $}
    \DisplayProof
    \end{center}
    
    \caption{Traversal step relation}
    \label{fig:traversal-step}
\end{figure}

\subsection{Simulation relation}

\subsubsection{Basic simulation}

\begin{definition}
  An pair of events $\tup{e, \hat{e}}$ where $e \in G.\lE$ and $\hat{e} \in S.\lE$
  is a \emph{forwarding pair} in
  the traversal configuration $\tup{C, I}$ of the \imm execution graph $G$,
  denoted as $\tup{e, \hat{e}} \in \forward(S, G, \tup{C, I}, f)$, if the following is true:
  \begin{enumerate}[label=\textbf{F.\arabic*}]

    \item \label{item:frwd-coverable}
      $G \vdash \tup{C, I} \trstep \tup{C \uplus \{e\}, I}$;
     
    \item \label{item:frwd-lab}
      $G.\lLAB(e) = S.\lLAB(\hat{e})$;

    \item \label{item:frwd-front}
      $S.\lPO;[\hat{e}] \subseteq [f(C)];S.\lPO$

    \item \label{item:frwd-rf}
      $\forall{w \in G.\lW} ~.~ \tup{w, e} \in G.\lRF \Rightarrow \tup{f(w), \hat{e}} \in S.\lRF$.

  \end{enumerate}
\end{definition}

\begin{definition}
  Relation $\simRb(S, G, \tup{C, I}, f)$, that binds an 
  event structure $S$, an \imm execution graph $G$,
  a traversal configuration $\tup{C, I}$
  and an event mapping $f : G.\lE \fun S.\lE$ 
  holds if the following conditions are met:
  \begin{enumerate}[label=\textbf{S.\arabic*}]

  \item \label{item:sim-lab}
    $\forall{e \in C \cup I} ~.~ G.\lLAB(e) = S.\lLAB(f(e))$;

  \item \label{item:sim-po-prfx} 
    $S.\lPO;[f(C)] \subseteq [f(C)];S.\lPO$;

  \item \label{item:sim-po}
    $f([C];G.\lPO;[C]) = [f(C)];S.\lPO;[f(C)]$;
    
  \item \label{item:sim-cf}
    $[f(C)];S.\lCF;[f(C)] = \emptyset$;

  \item \label{item:sim-rf}
    $f([I];G.\lRF;[C]) = [f(I)];S.\lRF;[f(C)]$;

  \item \label{item:sim-mo}
    $f([I];G.\lMO;[I]) = [f(I)];S.\lMO;[f(I)]$.
    
  \end{enumerate}
\end{definition}

\begin{lemma}
  \label{lemma:sim-forward}
  Given $S$, $G$, $\tup{C, I}$, $f$
  if the relation $\simRb(S, G, \tup{C, I}, f)$ holds and 
  there exists forwarding pair $\tup{e, \hat{e}} \in \forward(S, G, \tup{C, I}, f)$
  then $\simRb(S, G, \tup{C \uplus \{e\}, I}, f)$ holds. 
\end{lemma}

\begin{proof}
  Let us prove that all conditions for $\simRb(S, G, \tup{C \uplus \{e\}, I}, f)$ are met.

  \begin{itemize}

  \item \ref{item:sim-lab} \\
    \begin{equation}
      \begin{split}
        & \forall{a \in (C \uplus \{e\}) \cup I} ~.~ G.\lLAB(a) = S.\lLAB(s(a)) \iff \\
        & \forall{a \in C \cup I} ~.~
          G.\lLAB(a) = S.\lLAB(s(a)) \wedge G.\lLAB(e) = S.\lLAB(s(e))
      \end{split}
    \end{equation}
    First conjunct follows from the $\simRb(S, G, \tup{C, I}, s, t)$.
    Second conjunct follows from the \ref{item:frwd-descr}.

  \item \ref{item:sim-cf} \\
    We need to show that
    $\forall{a, b \in C \uplus \{e\}} ~.~ \tup{s(a), s(b)} \not\in S.\lCF$.
    Let us consider the opposite, that is
    $\exists{a, b \in C \uplus \{e\}} ~.~ \tup{s(a), s(b)} \in S.\lCF$.
    Giving that $\simRb(S, G, \tup{C, I}, s, t)$ holds,
    we can conclude that \mbox{$a = e \vee b = e$}.
    Let us suppose that $a = e$.
    Consider $c$ such that $\tup{c, e} \in G.\lPO_{imm}$.
    Because $e \in \coverable(G, C, I)$ we know that $c \in C$.
    From that and $\simRb(S, G, \tup{C, I}, s, t)$
    we can conclude that $\tup{s(b), s(c)} \not\in S.\lCF$.
    If $\tup{s(b), s(c)} \in S.\lPO$ then by transitivity of $S.\lPO$
    $\tup{s(b), s(e)} \in S.\lPO$
    which contradicts the fact that $\tup{s(b), s(e)} \in S.\lCF$.
    Otherwise \mbox{$\tup{s(c), s(b)} \in S.\lPO$}.
    From \ref{item:sim-po-imm} follows that $\tup{c, b} \in G.\lPO$
    and thus $\tup{e, b} \in G.\lPO$.
    But since $b \in C$ and thus $b \in \coverable(G, C, I)$,
    we can conclude that it shoud be the case that $e \in C$.
    But that contradicts $e \in \nextset(G, C, I)$.

  \item \ref{item:sim-po-prfx} \\

    \begin{equation*}
      \begin{split}
        &     S.\lPO;[s(C \uplus \{e\})] \subseteq [s(C \uplus \{e\})];S.\lPO \iff \\
        &\iff S.\lPO;[s(C)] \cup S.\lPO;[s(e)] \subseteq [s(C)];S.\lPO \cup [s(e)];S.\lPO
      \end{split}
    \end{equation*}

    $S.\lPO;[s(C)] \subseteq [s(C)];S.\lPO$ because $\simRb(S, G, \tup{C, I}, s, t)$ holds.
    Let us show that $S.\lPO;[s(e)] \subseteq [s(C)];S.\lPO$.
    Suppose there exists $\hat{a}$ s.t. $\tup{\hat{a}, s(e)} \in S.\lPO$ but
    $\hat{a} \not\in s(C)$.
    Because of \ref{item:frwd-front}
    it must be the case that $\tup{\hat{a}, s(\front(C, \lTID(e))} \in S.\lPO$.
    But $\front(C, \lTID(e)) \in C$ and thus from $\ref{item:sim-po-prfx}$
    of $\simRb(S, G, \tup{C, I}, s, t)$ follows that $\hat{a} \in s(C)$. Contradiction.

  \item \ref{item:sim-po} \\
    First, note that 
    
    \begin{equation*}
      \begin{split}
        [C \uplus \{e\}];G.\lPO;[C \uplus \{e\}] =
          [C];G.\lPO;[C] \cup [C];G.\lPO;[e] \cup [e];G.\lPO;[C]
      \end{split}
    \end{equation*}

    Also, as $e \in \nextset(G, C, I)$, $e \not\in C$ and
    because $C$ is $G.\lPO$ prefix closed, $[e];G.\lPO;[C]$ is empty.
    Thus we only have to show that

    \begin{equation*}
      \begin{split}
        & \forall{\tup{e_1, e_2} \in [C];G.\lPO;[C] \cup [C];G.\lPO;[e]} ~.~
            \tup{s(e_1), s(e_2)} \in S.\lPO \iff \\
        & \iff
          \forall{\tup{e_1, e_2} \in [C];G.\lPO;[C]} ~.~ \tup{s(e_1), s(e_2)} \in S.\lPO ~ \wedge \\
        & \wedge
          \forall{\tup{e_1, e_2} \in [C];G.\lPO;[e]} ~.~ \tup{s(e_1), s(e_2)} \in S.\lPO
      \end{split}
    \end{equation*}

    First conjunct follows from the $\simRb(S, G, \tup{C, I}, s, t)$.
    Let $c = \front(C, \lTID(\hat{e}))$. 
    From \ref{item:frwd-front} we know that $\tup{s(c), s(e)} \in S.\lPO_{imm}$.
    Thus, from \ref{item:sim-po-imm} of $\simRb(S, G, \tup{C, I}, s, t)$
    follows that $\tup{c, e} \in G.\lPO_{imm}$ and $c \in C$.
    From that and \ref{item:sim-po} of $\simRb(S, G, \tup{C, I}, s, t)$
    we can conclude that \\
    $\forall{\tup{e_1, e_2} \in [C];G.\lPO;[c]} ~.~ \tup{s(e_1), s(e_2)} \in S.\lPO$.
    The second conjunct follows directly from the later fact and the transitivity of $\lPO$.

  \item \ref{item:sim-rf} \\
    
    \begin{equation*}
      \begin{split}
        & \forall{\tup{w, r} \in [I];G.\lRF;[C \uplus \{e\}]} ~.~
          \tup{s'(w), s'(r)} \in S.\lRF \iff \\
        & \iff \forall{\tup{w, r} \in [I];G.\lRF;[C]} ~.~ \tup{s(w), s(r)} \in S.\lRF ~ \wedge \\
        & \wedge \forall{w \in I}. \tup{w, e} \in G.\lRF \Rightarrow \tup{s(w), \hat{e}} \in S.\lRF
      \end{split}
    \end{equation*}

    Left conjunct follows immediately from $\simRb(S, G, \tup{C, I}, s, t)$.
    Right conjunct follows from the \ref{item:frwd-rf}.
    
  \end{itemize}
  
\end{proof}

\begin{lemma}
  Given $S$, $G$, $\tup{C, I}$, $f$, $\Prog$ and $\Execs{\Prog}$
  s.t. $\Execs{\Prog} \vdash G$,
  if the relation $\simRb(S, G, \tup{C, I}, f)$ holds and
  there exists $e$ s.t. $G \vdash \tup{C, I} \trstep \tup{C \uplus \{e\}, I}$ then
  there exists $\hat{e}, S', f'$ such that
  $\Execs \vdash S \esstep{\hat{e}} S'$
  and $\simRb(S', G, \tup{C \uplus \{e\}, I}, f')$ holds.
\end{lemma}

\begin{proof}
  %% If we can show that $\simRb(S', G, \tup{C, I}, s', t')$ holds and
  %% $e \in \forward(S', G, \tup{C, I}, s', t')$ then by Lemma \ref{lemma:sim-forward}
  %% it follows that $\simRb(S', G, \tup{C \uplus \{e\}, I}, s', t')$ holds.

  %% First, let us show that $\simRb(S', G, \tup{C, I}, s', t')$ holds.

  %% \begin{itemize}

  %% \item \ref{item:sim-dom-s} and \ref{item:sim-dom-t} holds trivially.

  %% \item \ref{item:sim-ts-id}-\ref{item:sim-lab-wf} follows trivially from [BASIC-STEP].

  %% \item \ref{item:sim-cf}-\ref{item:sim-po} holds, because
  %%   $e \in \nextset(G, C, I)$,
  %%   %(thus $e \not\in C$ and $\forall{a \in C}~.~\tup{a, e} \in G.\lPO$),
  %%   $s'_{|C} = s_{|C}$ and $\simRb(S, G, \tup{C, I}, s, t)$.

  %% \item \ref{item:sim-po-imm} holds, because we have added to $S$
  %%   one single $\lPO_{imm}$ edge, that is $\tup{s(c), \hat{e}}$.
  %%   We also know that $t(\hat{e}) = e$ and since $e \in \nextset(G, C, I)$
  %%   there is a corresponding $\lPO_{imm}$ edge in $G$ ($\tup{c, e} \in G.\lPO_{imm}$).

  %% \item \ref{item:sim-rf} \\
  %%   If $e \not\in G.\lW$ then it holds trivially
  %%   (since in this case $s'_{|C \cup I} = s_{|C \cup I}$).
  %%   Otherwise consider $s(e)$ and $s'(e)$.
  %%   From [WRITE-STEP] we know that $\tup{s(e), s'(e)} \in S.\lEW$.
  %%   This leads to 
  %%   $\forall{\tup{w, r} \in [e];G.\lRF;[C]} ~.~ \tup{s'(w), s'(r)} \in S.\lJF$,
  %%   and thus \ref{item:sim-rf} holds.

  %% \item \ref{item:sim-ew} \\
  %%   If $e \not\in G.\lW$ then it holds trivially.
  %%   Otherwise, from \ref{item:sim-ew} of $\simRb(S, G, \tup{C, I}, s, t)$ we know that
  %%   $\forall{\hat{w_1}, \hat{w_2} \in S.\lW} ~.~ t(\hat{w_1}) = t(\hat{w_2}) = e
  %%   \Rightarrow \tup{\hat{w_1}, \hat{w_2}} \in S.\lEW$.
  %%   Also, we know that $e \in I$ (because $e \in G.\lW \cap \coverable(G, C, I)$)
  %%   and thus $t(s(e)) = e$.
  %%   From [WRITE-STEP] $\tup{s(e), s'(e)} \in S.\lEW$ and thus
  %%   $\forall{\hat{w} \in S.\lW}~.~ t(\hat{w}) = e \Rightarrow \tup{\hat{w}, s'(e)} \in S.\lEW$.
  %%   That means \ref{item:sim-ew} holds.
  
  %% \end{itemize}

  %% Now, let us show that $e \in \forward(S', G, \tup{C, I}, s', t')$.

  %% \begin{itemize}

  %%   \item \ref{item:frwd-coverable} is a Lemma primise;

  %%   \item \ref{item:frwd-dom-s}, \ref{item:frwd-ts-id} and \ref{item:frwd-descr}
  %%     follows trivially from [BASIC-STEP];

  %%   \item \ref{item:frwd-front}
  %%     $\tup{s(\front(C, \lTID(\hat{e}))), s(e)} \in S.\lPO_{imm}$;

  %%   \item \ref{item:frwd-rf}
  %%     $\tup{\hat{w}, s(e)} \in S.\lRF \Leftrightarrow \tup{t(\hat{w}), e} \in G.\lRF$.
    
  %% \end{itemize}
\end{proof}

\subsubsection{Weak simulation}

\begin{definition}
  Relation $\simRw(S, G, \tup{C, I}, f, F)$, that binds an 
  event structure $S$, an \imm execution graph $G$,
  a traversal configuration $\tup{C, I}$,
  a source $s : G.\lE \fun S.\lE$ and target $t : S.\lE \fun G.\lE$ event mappings,
  a program counter $\lPC$
  holds whenever $\simRb(S, G, \tup{C, I}, s)$ holds and
  additionally the following conditions are met:
  \begin{enumerate}[label=\textbf{S.\arabic*},start=6]

    \item \label{item:sim-pc}
      $\forall{i \in \Tid} ~.~ \lPC(i) \in C$

     
    \item \label{item:sim-vis-weak}
      $\forall{e \in (C \cup I)} ~.~
      e \in \dom{G.\lPO^?;[\lPC(\lTID(e))]} \implies s(e) \in \lVIS(S)$.

  \end{enumerate}
\end{definition}

\begin{lemma}
  Given $\lES$, $G$, $\tup{\lC, I}$, $s$, $t$ and $\lPC$
  if the relation $\simRw(S, G, \tup{\lC, I}, s, t, \lPC)$ holds and 
  there exists $\tup{\lC', I'}$
  s.t. $G \vdash \tup{\lC, I} \rightarrow \tup{\lC', I'}$
  then there exists $S'$, $s'$ and $t'$ s.t. $G \vdash (s, t, S) \ESstepstar (s', t', S')$
  and $\simRw(S', G, \tup{\lC', I'}, s', t', \lPC)$ 
\end{lemma}

\begin{proof}

\end{proof}

\subsubsection{Simulation}

\begin{definition}
  Given $S$, $G$, $\tup{C, I}$, $f$ a relation $\simR(S, G, \tup{C, I}, f)$ holds
  whenever $\simRb(S, G, \tup{C, I}, f)$ holds and additionally
  the following conditions are met:
  \begin{enumerate}[label=\textbf{S.\arabic*},start=7]
    \item \label{item:sim-vis}
       $f(C \cup I) \subseteq \lVIS(S)$
  \end{enumerate}
\end{definition}

\begin{lemma}
  Given $S$, $G$, $\tup{C, I}$, $f$, $\Prog$ and $\Execs{\Prog}$
  s.t. $G \vdash \Execs{\Prog}$,
  if the relation $\simR(S, G, \tup{C, I}, f)$ holds
  and there exists $e$ s.t. $G \vdash \tup{C, I} \trstep \tup{C', I'}$
  then there exists $S'$ and $f'$ s.t.
  $\Execs{\Prog} \vdash S \esstepstar S'$ and
  $\simR(S, G, \tup{C, I}, f)$ holds.
\end{lemma}

\begin{proof}
  
\end{proof}

\begin{lemma}
  Given $S$, $G$, $\tup{C, I}$, $f$, $\Prog$ and $\Execs{\Prog}$
  s.t. $\Execs{\Prog} \vdash G$ and $\simR(S, G, \tup{C, I}, f)$ hold
  then there exists $G^{\cert} \in \certify(\Execs{\Prog}, G, C, I)$ s.t.
  $S \rhd G^{\cert}$ holds.
\end{lemma}

\begin{proof}
  
\end{proof}
  
\setmonofont[Mapping=tex-text]{CMU Typewriter Text}
\bibliographystyle{acm}
\bibliography{main.bib}

\end{document}
